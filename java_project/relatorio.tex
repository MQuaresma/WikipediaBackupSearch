\documentclass[a4paper,12pt,portuguese]{report}
\usepackage[utf8]{inputenc} 
\usepackage[portuguese]{babel}
\usepackage{ragged2e}
\usepackage{fancyvrb} 

\title{Projeto de LI-3}
\author{Martins, José(a78821)\
        \and
        Costa, Mariana(a78824)\
        \and
        Quaresma, Miguel(a77049)
        }         
\date{\today}

\begin{document}

\begin{titlepage}
\maketitle
\end{titlepage}

\tableofcontents

\chapter{Introdução}

\chapter{Desenvolvimento}

\section{Bibliotecas}
Neste trabalho usamos várias bibliotecas disponiveis pelo Java, sendo de destacar as bibliotecas pertencentes ao StAX no qual nos permitiu realizar o parse dos ficheiros XML percorrendo-os aos poucos em vez de realizar o parse através de um ficheiro guardado em memória com toda a informação importante do ficheiro XML poupando assim tanto em recursos como em tempo de execução. 

\section{Classes}
Sentimos a necessidade de criar duas classes principais, sendo que uma representa artigos e outra representa contribuidores. Portanto destas classes podemos construir quantas instancias artigos ou contribuintes quisermos. 

\subsection{Article.java}

\subsection{Contributor.java}


Para além disso temos mais 3 classes nas quais implementamos comparadores necessarios para implementar algumas queries. 

\subsection{ArtCompareText.java}

\subsection{ArtCompareWords.java}

\subsection{ComparatorContributorRevs.java}

Por fim a classe "mãe". 

\subsection{QueryEngineImpl.java}
Possui um HashMap de artigos e um TreeMap de contribuintes de modo a agregar artigos num conjunto e contribuidores também num conjunto, sobre os quais podemos aplicar métodos. Ainda nessa classe possuímos dois longs de modo a guardar os artigos únicos(artUn) e os artigos totais(artTot).

\chapter{Conclusão}

\end{document}
