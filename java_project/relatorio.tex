\documentclass[a4paper,11pt]{article}
\usepackage[utf8]{inputenc} 
\usepackage[portuges]{babel}
\usepackage{enumitem}
\usepackage{ragged2e}
\usepackage{fancyvrb} 

\title{Projeto de LI-3}
\author{Martins, José(a78821)\
        \and
        Costa, Mariana(a78824)\
        \and
        Quaresma, Miguel(a77049)
        }         
\date{\today}

\begin{document}

\begin{titlepage}
\maketitle
\end{titlepage}

\tableofcontents
\newpage

\section{Introdução}
A \textbf{Wikipédia} é uma  das muitas fontes de informação disponíveis na Web, possuindo uma quantidade bastante considerável de dados/informação.
No entanto para que esta informação seja útil é necessário saber processá-la e acima de tudo fazê-lo em tempo útil, sendo para isso necessário recorrer a estruturas de dados que permitam uma pesquisa rápida mas sem uso excessivo de memória. O objetivo deste projeto é, por isso, implementar uma aplicação (em \textit{Java}) que responde a um conjunto de \textit{queries} relativas a \textit{snapshots} da \textbf{Wikipédia} em tempo reduzido e de uma forma correta, recorrendo para isso a um conjunto de recursos já existentes e a estruturas de dados que permitam atingir este objetivo.

\section{Desenvolvimento}

\subsection{Bibliotecas}
Neste trabalho usamos várias bibliotecas disponibilizadas pelo Java, sendo de destacar as bibliotecas pertencentes ao StAX no qual nos permitiu realizar o parse dos ficheiros XML percorrendo-os aos poucos em vez de realizar o parse através de um ficheiro guardado em memória com toda a informação importante do ficheiro XML poupando assim tanto em recursos como em tempo de execução. 

\subsection{Classes}
Sentimos a necessidade de criar duas classes principais, sendo que uma representa artigos e outra representa contribuidores. Portanto destas classes podemos construir quantas instancias artigos ou contribuintes quisermos. 

\subsubsection{Article.java}
A classe article, seguindo a linha do trabalho anterior e de modo a responder às queries propostas possui as seguintes variáveis de instancia:
\begin{verbatim}
    private long id; //id do artigo
    private String title; //título do artigo
    private Map<Long,String> revisions; // revisões do artigo
                                        // chave: id da revisão
                                        // valor: timestamp da revisão
    private long nRev; //número de revisões
    private long len; //das revisões, o número de caracteres 
                      //da revisão com mais caracteres 
    private long words; //das revisões, o número de palavras 
                        //da revisão com mais palavras
\end{verbatim}
Nesta classe, para além dos métodos usuais, é de destacar o método \texttt{public void setNewLenghtWords(String text)} visto ser este o  método que calcula o número de palavras e caractéres de uma String passada como argumento e caso os valores sejam maiores que os existentes nas variáveis de instância \texttt{len} e \texttt{words} atualiza-os.

\subsubsection{Contributor.java}
Quanto à classe contributor, como na classe article seguimos as directrizes do trabalho anterior e como tal possui as seguintes variáveis de instancia:
\begin{verbatim}
    private long id; //id do contribuidor
    private String name; //nome do contribuidor
    private int nRev; // número de revisões do contribuidor
\end{verbatim}
Nesta classe, foram apenas implementados os métodos usuais (gets, sets, clone, etc).

\subsection{Comparators}
Para além das classes já referidas estão presentes mais 3 classes de comparadores necessários para implementar algumas queries, sendo esses comparadores os seguintes:
\begin{itemize}[align=left]
\item[\texttt{ArtCompareText.java}]: compara dois artigos devolvendo -1 caso o 1º argumento tenha mais caracteres ou caso tenha o mesmo número de caracteres que o 2º argumento, mas tenho um menor id, caso contrário devolve 1;
\item[\texttt{ArtCompareWords.java}]: compara dois artigos devolvendo -1 caso o 1    º argumento tenha mais palavras ou caso tenha o mesmo número de palavras que o 2º argumento, mas tenho um menor id, caso contrário devolve 1;
\item[\texttt{ComparatorContributorRevs.java}]: compara dois contribuidores devolvendo -1 caso o 1º argumento tenha mais revisões ou caso tenha o mesmo número de revisões que o 2º argumento, mas tenho um menor id, caso contrário devolve 1;
\end{itemize}

\subsection{QueryEngineImpl.java}
Possui um HashMap de artigos e um TreeMap de contribuintes de modo a agregar artigos num conjunto e contribuidores também num conjunto, sobre os quais podemos aplicar métodos. Ainda nessa classe possuímos dois longs de modo a guardar os artigos únicos(artUn) e os artigos totais(artTot).

\section{Conclusão}

\end{document}
